\documentclass{article}

\usepackage[utf8]{inputenc}
% \usepackage[LGR, T1]{fontenc}

% \usepackage[greek]{babel} % This automatically transcribes letters. Use \textlatin to escape it
\usepackage{amsmath, amsfonts, amssymb}
\usepackage{amsthm}
% \usepackage{bbm}
% \usepackage{alphabeta}
% \usepackage{graphicx}
% \graphicspath{ {../output/} }
% \usepackage{hyperref}



% To make custom snippets, Ctrl+shift+P --> Preferences: Configure User Snippets -> latex (or latex.json)

% \newcommand{\inner}[2]{\left\langle #1 \mathrel{,} #2 \right\rangle}
% \newcommand{\norm}[1]{\left\| #1 \right\|}
% \newcommand{\T}[1]{{#1}^{\top}}  % alternatively, use {#1}^{\intercal} or {#1}'
% \newcommand{\ve}[1]{\boldsymbol{#1}}


\title{Algorithmic Data Science - Exercises Series 2}
\author{
    Konstantinos Papadakis\\
    Data Science and Machine Learning 03400149\\
    konstantinospapadakis@mail.ntua.gr
}
\date{\today}


\begin{document}

\maketitle

\newpage
%%%%%%%%%%%%%%%%%%%%%%%%%
% EXERCISE 1
%%%%%%%%%%%%%%%%%%%%%%%%%
\section*{Exercise 1}

%%%%%%%%%%%%%%%%%%%%%%%%%
% 1(a)
%%%%%%%%%%%%%%%%%%%%%%%%%
\subsubsection*{(a)}
We have that 
\begin{align*}
    &h_{a,b}(x) = h_{a,b}(y)\\
    \iff& ax + b \equiv ay + b \mod{m}\\
    \iff& a(x-y) \equiv 0 \mod{m}
\end{align*}
which in the case of \(x=m, \  y = 0\) is true \(\forall a, b\)
therefore
\[P(h_{a,b}(x) = h_{a,b}(y)) = 1 > \frac{1}{m}\]
meaning that the family is not universal.

%%%%%%%%%%%%%%%%%%%%%%%%%
% 1(b)
%%%%%%%%%%%%%%%%%%%%%%%%%
\subsection*{(b)}

This exercise is \emph{Theorem 11.5} in the book \emph{Introduction to Algorithms by Cormen et al.}.

Let \(x, y \in \mathbb{Z}_p: x \neq y\).

Define
\begin{align*}
    u &:= ax + b \mod{p}\\
    v &:= ay + b \mod{p}
\end{align*}

Note that \(u \neq v\)
since \(u - v \equiv a (x - y) \not\equiv 0 \mod{p}\)
because \(a \neq 0\) and \(x \not\equiv y \mod{p}\),
the later holding because by hypothesis \(x \neq y \ \textrm{and}\ x,y < p\)
Therefore, there are no collisions when we apply \(x \mapsto ax + b \mod{p}\).

We proceed to show that
\((a, b) \mapsto (ax + b \mod{p},\ ax + b \mod{p})\) is a bijection between
the pairs \((a, b) \in \mathbb{Z}_p^* \times \mathbb{Z}_p\)
and the pairs \((u, v) \in \mathbb{Z}_p \times \mathbb{Z}_p: u \neq v\).

We can solve for \(a, b\) and get a unique solution
\begin{align*}
    a &= \frac{u-v}{x-y} \mod{p}\\
    b &= r - ak \mod{p}
\end{align*}
Where \(\frac{1}{t}\) is the inverse of \(t\) in \(\mathbb{Z}_p\)

Therefore the mapping is one to one.
Since we also have that both the domain and the codomain have \(p(p-1)\) elements,
the mapping is a bijection.
Thus, if \((a, b)\) is uniformely distributed, so is \((u, v)\).

Therefore, the probability that \(x, y \in \mathbb{Z}_p: x \neq y\) collide
is equal to the probability that \(u \equiv v \mod{m}\) collide
when \((u, v) \in \mathbb{Z}_p \times \mathbb{Z}_p: u \neq v\) are chosen uniformely randomly.
We proceed to calculate that probability.

Given \(u\), of the \(p-1\) possible remaining values for \(v\) we have that
at most \(\lceil{\frac{p}{m}}\rceil - 1 \leq \frac{p-1}{m}\)
can collide with \(u\). 

Therefore the probability of colision is \(\leq \frac{1}{m}\),
meaning that the hash function family is universal.

%%%%%%%%%%%%%%%%%%%%%%%%%
% 1(c)
%%%%%%%%%%%%%%%%%%%%%%%%%
\subsubsection*{(c)}\label{1c}

The proof in \ref{1c} is still valid, since \(x \in U \implies x < p\) is still valid.
Therefore the hash function family remains universal.


%%%%%%%%%%%%%%%%%%%%%%%%%
% EXERCISE 2
%%%%%%%%%%%%%%%%%%%%%%%%%
\section*{Exercise 2}



\end{document}